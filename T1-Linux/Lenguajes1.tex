%%%%%%%%%%%%%%%%%%%%%%%%%%%%%%%%%%%%%%%%%%%%%%%%%%%%%%%%%%%%%%%%%%%%%%%%%%%%%%%%%
\begin{frame}
  \frametitle{Lenguajes}
      \begin{defi}
			Un \textbf{lenguaje} (formal) es un conjunto de cadenas
			\end{defi}
%%%%%%%%%%%%%%%%%%%%%%%%%%%%%%%%%%%%%%%%%

\begin{block}{Ejemplos:}
\begin{itemize}[<+->]
\item $L_1 = \{1, 234, 912, 456\}$ es un lenguaje sobre $\Sigma = \{0, 1, 2, \ldots, 9\}$
\item El conjunto de palabras correctas en castellano es un lenguaje sobre el alfabeto latino
\item El conjunto de programas correctos escritos en C
\end{itemize}
\end{block}

\begin{itemize}[<+->]
\item Si $\Sigma$ es un alfabeto, también es un lenguaje
\item Los lenguajes pueden ser infinitos: $L = \{a, aa, aaa, aaaa, aaaaa, \ldots \}$
Este lenguaje es infinito, a pesar de que todas sus cadenas tienen longitud finita
\item Cuando el cardinal de un lenguaje es grande, resulta difícil especificar qué palabras lo componen
\item El lenguaje vacío, $L = \emptyset$ es un lenguaje
\end{itemize}
\end{frame}
%%%%%%%%%%%%%%%%%%%%%%%%%%%%%%%%%%%%%%%%%%%%%%%%%%%%%%%%%%%%%%%%%%%%%%%%%%%%%%%%%
