%%%%%%%%%%%%%%%%%%%%%%%%%%%%%%%%%%%%%%%%%%%%%%%%%%%%%%%%%%%%%%%%%%%%%%%%%%%%%%%%%
\begin{frame}
  \frametitle{Subsecuencias}
      %%%%%%%%%%%%%%%%%%%%%%%%%%%%%%%%%%%%%%%%%
      \begin{block}{Definición: Sean $x, y \in \Sigma^*$}
			$y$ es una \textbf{subsecuencia} de $x$ si $y$ tiene símbolos de $x$ respetando su orden, pero no necesariamente contiguos
      \end{block}
      \pause

      \begin{block}{}
           \begin{itemize}[<+->]
           \item $x = x_1x_2 \ldots x_N$
					 \item $y = x_{i1} x_{i2} \ldots x_{ik}$
					 \item $1 \leq i1 \leq i2 \leq \ldots \leq ik \leq im$
           \end{itemize}
			\end{block}

      \pause
      \begin{block}{Ejemplo}
           \begin{itemize}[<+->]
           \item Sea $w = abracadabra$. Algunas subsecuencias de $w$ son:
					 \item $arda, rcdr, aaaa$
					 \item $\epsilon$ es subsecuencia de toda cadena
			     \item Toda subcadena es subsecuencia, pero el recíproco no es cierto
			     \item Ejercicio: ¿cuál es el número de subsecuencias de $x \in \Sigma^*$ si $|x|=n$?
           \end{itemize}
			\end{block}
\end{frame}
%%%%%%%%%%%%%%%%%%%%%%%%%%%%%%%%%%%%%%%%%%%%%%%%%%%%%%%%%%%%%%%%%%%%%%%%%%%%%%%%%
